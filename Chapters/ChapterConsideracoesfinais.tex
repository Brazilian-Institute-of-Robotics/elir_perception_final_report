\chapter{Conclusão}
\label{chap:conc}

No presente trabalho, foi apresentada um modelo de sistema de Percepção para robôs de inspeção de linhas de alta tensão, no qual fundamentos de processamento de imagem para detecção de pontos quentes são apresentados mais detalhadamente.

O trabalho do proposto além de se apresentar como uma contribuição para o desenvolvimento em robótica móvel com o objetivo de redução do trabalho humano em atividades de risco também se mostra como ferramenta de aproximação dos cursos de graduação nos projetos de pesquisa e desenvolvimento em robótica.

Ao longo do trabalho foi proposto um algoritmo de detecção de pontos quentes através da binarização de imagem e busca de contornos, esta técnica apresentou resultados satisfatórios para a funcionalidade de Detecção. Entretanto, para a consideração de mais formatos de objeto e perfis de temperaturas diferenciados, outros métodos de processamento podem ser analisados em trabalhos futuros para se tornar uma solução mais completa e robusta.

Em relação a interface gráfica desenvolvida no projeto, ela é capaz de fornecer informações em tempo real ao operador facilitando tomadas de decisão e planejamento das atividades de manutenção. No trabalho, não foi implementado a tela com informações dos atuadores, podendo ser implementados em trabalhos futuros para apresentar uma solução mais completa de inspeção.

Além dos sistemas de desenvolvidos, o próprio planejamento e método utilizado no projeto podem ser considerados um resultado para auxílio de gestão de projetos robóticos. A equipe buscou utilizar as ferramentas mais utilizadas na área de robótica, implementando boas práticas de versionamento de código e de desenvolvimento.
Os suportes mecânicos para os sensores do sistema de Percepção apresentaram uma solução satisfatória para integração dos mesmos na parte mecânica do robô. Contudo, o projeto de suportes mecânicos que melhorem a manutenabilidade são desejados em próximos projetos.

Verifica-se, portanto, que os resultados obtidos no presente trabalho são considerados favoráveis e satisfatórios, bem como uma proposta de integração entre graduação e projetos de robótica, incentivando o desenvolvimento tecnológico e inovação. Os autores possuem interesse que os sistemas propostos sejam avaliados em ambiente de campo para que novos cenários sejam estudados bem como a implementação de novos sistemas ou aperfeiçoamento dos existentes.
