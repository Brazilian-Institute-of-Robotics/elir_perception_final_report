\begin{thesisresumo}
A manutenção de linhas de alta tensão além de ser uma atividade de alto custo é uma prática de alto risco a integridade física do operador. De modo a substituir o trabalho humano em atividades de risco, soluções em robótica estão cada vez mais frequentes por conta da confiabilidade empregada. O Electrical Line Inspection Robot (ELIR) é um robô para inspeção em linhas de alta tensão através inspeção térmica, o seu sistema de Percepção conta com uma série de sensores e é capaz de disponibilizar ao usuário final todas as ocorrências realizadas durante a missão bem como data, horário e localização. Este trabalho de conclusão de curso conta com a participação de estudantes de mestrado e doutorado da instituição a fim de promover pesquisa e desenvolvimeno na área de robótica. Toda a metodologia, conceito e resultados obtidos no desenvolvimento do sistema de percepção do robô estão descritos ao longo desse trabalho.

\ \\

% use de três a cinco palavras-chave

\textbf{Palavras-chave}: Linhas de transmissão, Inspeção, Robótica, Pontos quentes

\end{thesisresumo}
